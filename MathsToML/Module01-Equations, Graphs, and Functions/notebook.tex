
% Default to the notebook output style

    


% Inherit from the specified cell style.




    
\documentclass[11pt]{article}

    
    
    \usepackage[T1]{fontenc}
    % Nicer default font (+ math font) than Computer Modern for most use cases
    \usepackage{mathpazo}

    % Basic figure setup, for now with no caption control since it's done
    % automatically by Pandoc (which extracts ![](path) syntax from Markdown).
    \usepackage{graphicx}
    % We will generate all images so they have a width \maxwidth. This means
    % that they will get their normal width if they fit onto the page, but
    % are scaled down if they would overflow the margins.
    \makeatletter
    \def\maxwidth{\ifdim\Gin@nat@width>\linewidth\linewidth
    \else\Gin@nat@width\fi}
    \makeatother
    \let\Oldincludegraphics\includegraphics
    % Set max figure width to be 80% of text width, for now hardcoded.
    \renewcommand{\includegraphics}[1]{\Oldincludegraphics[width=.8\maxwidth]{#1}}
    % Ensure that by default, figures have no caption (until we provide a
    % proper Figure object with a Caption API and a way to capture that
    % in the conversion process - todo).
    \usepackage{caption}
    \DeclareCaptionLabelFormat{nolabel}{}
    \captionsetup{labelformat=nolabel}

    \usepackage{adjustbox} % Used to constrain images to a maximum size 
    \usepackage{xcolor} % Allow colors to be defined
    \usepackage{enumerate} % Needed for markdown enumerations to work
    \usepackage{geometry} % Used to adjust the document margins
    \usepackage{amsmath} % Equations
    \usepackage{amssymb} % Equations
    \usepackage{textcomp} % defines textquotesingle
    % Hack from http://tex.stackexchange.com/a/47451/13684:
    \AtBeginDocument{%
        \def\PYZsq{\textquotesingle}% Upright quotes in Pygmentized code
    }
    \usepackage{upquote} % Upright quotes for verbatim code
    \usepackage{eurosym} % defines \euro
    \usepackage[mathletters]{ucs} % Extended unicode (utf-8) support
    \usepackage[utf8x]{inputenc} % Allow utf-8 characters in the tex document
    \usepackage{fancyvrb} % verbatim replacement that allows latex
    \usepackage{grffile} % extends the file name processing of package graphics 
                         % to support a larger range 
    % The hyperref package gives us a pdf with properly built
    % internal navigation ('pdf bookmarks' for the table of contents,
    % internal cross-reference links, web links for URLs, etc.)
    \usepackage{hyperref}
    \usepackage{longtable} % longtable support required by pandoc >1.10
    \usepackage{booktabs}  % table support for pandoc > 1.12.2
    \usepackage[inline]{enumitem} % IRkernel/repr support (it uses the enumerate* environment)
    \usepackage[normalem]{ulem} % ulem is needed to support strikethroughs (\sout)
                                % normalem makes italics be italics, not underlines
    

    
    
    % Colors for the hyperref package
    \definecolor{urlcolor}{rgb}{0,.145,.698}
    \definecolor{linkcolor}{rgb}{.71,0.21,0.01}
    \definecolor{citecolor}{rgb}{.12,.54,.11}

    % ANSI colors
    \definecolor{ansi-black}{HTML}{3E424D}
    \definecolor{ansi-black-intense}{HTML}{282C36}
    \definecolor{ansi-red}{HTML}{E75C58}
    \definecolor{ansi-red-intense}{HTML}{B22B31}
    \definecolor{ansi-green}{HTML}{00A250}
    \definecolor{ansi-green-intense}{HTML}{007427}
    \definecolor{ansi-yellow}{HTML}{DDB62B}
    \definecolor{ansi-yellow-intense}{HTML}{B27D12}
    \definecolor{ansi-blue}{HTML}{208FFB}
    \definecolor{ansi-blue-intense}{HTML}{0065CA}
    \definecolor{ansi-magenta}{HTML}{D160C4}
    \definecolor{ansi-magenta-intense}{HTML}{A03196}
    \definecolor{ansi-cyan}{HTML}{60C6C8}
    \definecolor{ansi-cyan-intense}{HTML}{258F8F}
    \definecolor{ansi-white}{HTML}{C5C1B4}
    \definecolor{ansi-white-intense}{HTML}{A1A6B2}

    % commands and environments needed by pandoc snippets
    % extracted from the output of `pandoc -s`
    \providecommand{\tightlist}{%
      \setlength{\itemsep}{0pt}\setlength{\parskip}{0pt}}
    \DefineVerbatimEnvironment{Highlighting}{Verbatim}{commandchars=\\\{\}}
    % Add ',fontsize=\small' for more characters per line
    \newenvironment{Shaded}{}{}
    \newcommand{\KeywordTok}[1]{\textcolor[rgb]{0.00,0.44,0.13}{\textbf{{#1}}}}
    \newcommand{\DataTypeTok}[1]{\textcolor[rgb]{0.56,0.13,0.00}{{#1}}}
    \newcommand{\DecValTok}[1]{\textcolor[rgb]{0.25,0.63,0.44}{{#1}}}
    \newcommand{\BaseNTok}[1]{\textcolor[rgb]{0.25,0.63,0.44}{{#1}}}
    \newcommand{\FloatTok}[1]{\textcolor[rgb]{0.25,0.63,0.44}{{#1}}}
    \newcommand{\CharTok}[1]{\textcolor[rgb]{0.25,0.44,0.63}{{#1}}}
    \newcommand{\StringTok}[1]{\textcolor[rgb]{0.25,0.44,0.63}{{#1}}}
    \newcommand{\CommentTok}[1]{\textcolor[rgb]{0.38,0.63,0.69}{\textit{{#1}}}}
    \newcommand{\OtherTok}[1]{\textcolor[rgb]{0.00,0.44,0.13}{{#1}}}
    \newcommand{\AlertTok}[1]{\textcolor[rgb]{1.00,0.00,0.00}{\textbf{{#1}}}}
    \newcommand{\FunctionTok}[1]{\textcolor[rgb]{0.02,0.16,0.49}{{#1}}}
    \newcommand{\RegionMarkerTok}[1]{{#1}}
    \newcommand{\ErrorTok}[1]{\textcolor[rgb]{1.00,0.00,0.00}{\textbf{{#1}}}}
    \newcommand{\NormalTok}[1]{{#1}}
    
    % Additional commands for more recent versions of Pandoc
    \newcommand{\ConstantTok}[1]{\textcolor[rgb]{0.53,0.00,0.00}{{#1}}}
    \newcommand{\SpecialCharTok}[1]{\textcolor[rgb]{0.25,0.44,0.63}{{#1}}}
    \newcommand{\VerbatimStringTok}[1]{\textcolor[rgb]{0.25,0.44,0.63}{{#1}}}
    \newcommand{\SpecialStringTok}[1]{\textcolor[rgb]{0.73,0.40,0.53}{{#1}}}
    \newcommand{\ImportTok}[1]{{#1}}
    \newcommand{\DocumentationTok}[1]{\textcolor[rgb]{0.73,0.13,0.13}{\textit{{#1}}}}
    \newcommand{\AnnotationTok}[1]{\textcolor[rgb]{0.38,0.63,0.69}{\textbf{\textit{{#1}}}}}
    \newcommand{\CommentVarTok}[1]{\textcolor[rgb]{0.38,0.63,0.69}{\textbf{\textit{{#1}}}}}
    \newcommand{\VariableTok}[1]{\textcolor[rgb]{0.10,0.09,0.49}{{#1}}}
    \newcommand{\ControlFlowTok}[1]{\textcolor[rgb]{0.00,0.44,0.13}{\textbf{{#1}}}}
    \newcommand{\OperatorTok}[1]{\textcolor[rgb]{0.40,0.40,0.40}{{#1}}}
    \newcommand{\BuiltInTok}[1]{{#1}}
    \newcommand{\ExtensionTok}[1]{{#1}}
    \newcommand{\PreprocessorTok}[1]{\textcolor[rgb]{0.74,0.48,0.00}{{#1}}}
    \newcommand{\AttributeTok}[1]{\textcolor[rgb]{0.49,0.56,0.16}{{#1}}}
    \newcommand{\InformationTok}[1]{\textcolor[rgb]{0.38,0.63,0.69}{\textbf{\textit{{#1}}}}}
    \newcommand{\WarningTok}[1]{\textcolor[rgb]{0.38,0.63,0.69}{\textbf{\textit{{#1}}}}}
    
    
    % Define a nice break command that doesn't care if a line doesn't already
    % exist.
    \def\br{\hspace*{\fill} \\* }
    % Math Jax compatability definitions
    \def\TeX{\mbox{T\kern-.14em\lower.5ex\hbox{E}\kern-.115em X}}
    \def\LaTeX{\mbox{L\kern-.325em\raise.21em\hbox{$\scriptstyle{A}$}\kern-.17em}\TeX}
    \def\gt{>}
    \def\lt{<}
    % Document parameters
    \title{01-01-Introduction to Equations}
    
    
    

    % Pygments definitions
    
\makeatletter
\def\PY@reset{\let\PY@it=\relax \let\PY@bf=\relax%
    \let\PY@ul=\relax \let\PY@tc=\relax%
    \let\PY@bc=\relax \let\PY@ff=\relax}
\def\PY@tok#1{\csname PY@tok@#1\endcsname}
\def\PY@toks#1+{\ifx\relax#1\empty\else%
    \PY@tok{#1}\expandafter\PY@toks\fi}
\def\PY@do#1{\PY@bc{\PY@tc{\PY@ul{%
    \PY@it{\PY@bf{\PY@ff{#1}}}}}}}
\def\PY#1#2{\PY@reset\PY@toks#1+\relax+\PY@do{#2}}

\expandafter\def\csname PY@tok@gd\endcsname{\def\PY@tc##1{\textcolor[rgb]{0.63,0.00,0.00}{##1}}}
\expandafter\def\csname PY@tok@gu\endcsname{\let\PY@bf=\textbf\def\PY@tc##1{\textcolor[rgb]{0.50,0.00,0.50}{##1}}}
\expandafter\def\csname PY@tok@gt\endcsname{\def\PY@tc##1{\textcolor[rgb]{0.00,0.27,0.87}{##1}}}
\expandafter\def\csname PY@tok@gs\endcsname{\let\PY@bf=\textbf}
\expandafter\def\csname PY@tok@gr\endcsname{\def\PY@tc##1{\textcolor[rgb]{1.00,0.00,0.00}{##1}}}
\expandafter\def\csname PY@tok@cm\endcsname{\let\PY@it=\textit\def\PY@tc##1{\textcolor[rgb]{0.25,0.50,0.50}{##1}}}
\expandafter\def\csname PY@tok@vg\endcsname{\def\PY@tc##1{\textcolor[rgb]{0.10,0.09,0.49}{##1}}}
\expandafter\def\csname PY@tok@vi\endcsname{\def\PY@tc##1{\textcolor[rgb]{0.10,0.09,0.49}{##1}}}
\expandafter\def\csname PY@tok@vm\endcsname{\def\PY@tc##1{\textcolor[rgb]{0.10,0.09,0.49}{##1}}}
\expandafter\def\csname PY@tok@mh\endcsname{\def\PY@tc##1{\textcolor[rgb]{0.40,0.40,0.40}{##1}}}
\expandafter\def\csname PY@tok@cs\endcsname{\let\PY@it=\textit\def\PY@tc##1{\textcolor[rgb]{0.25,0.50,0.50}{##1}}}
\expandafter\def\csname PY@tok@ge\endcsname{\let\PY@it=\textit}
\expandafter\def\csname PY@tok@vc\endcsname{\def\PY@tc##1{\textcolor[rgb]{0.10,0.09,0.49}{##1}}}
\expandafter\def\csname PY@tok@il\endcsname{\def\PY@tc##1{\textcolor[rgb]{0.40,0.40,0.40}{##1}}}
\expandafter\def\csname PY@tok@go\endcsname{\def\PY@tc##1{\textcolor[rgb]{0.53,0.53,0.53}{##1}}}
\expandafter\def\csname PY@tok@cp\endcsname{\def\PY@tc##1{\textcolor[rgb]{0.74,0.48,0.00}{##1}}}
\expandafter\def\csname PY@tok@gi\endcsname{\def\PY@tc##1{\textcolor[rgb]{0.00,0.63,0.00}{##1}}}
\expandafter\def\csname PY@tok@gh\endcsname{\let\PY@bf=\textbf\def\PY@tc##1{\textcolor[rgb]{0.00,0.00,0.50}{##1}}}
\expandafter\def\csname PY@tok@ni\endcsname{\let\PY@bf=\textbf\def\PY@tc##1{\textcolor[rgb]{0.60,0.60,0.60}{##1}}}
\expandafter\def\csname PY@tok@nl\endcsname{\def\PY@tc##1{\textcolor[rgb]{0.63,0.63,0.00}{##1}}}
\expandafter\def\csname PY@tok@nn\endcsname{\let\PY@bf=\textbf\def\PY@tc##1{\textcolor[rgb]{0.00,0.00,1.00}{##1}}}
\expandafter\def\csname PY@tok@no\endcsname{\def\PY@tc##1{\textcolor[rgb]{0.53,0.00,0.00}{##1}}}
\expandafter\def\csname PY@tok@na\endcsname{\def\PY@tc##1{\textcolor[rgb]{0.49,0.56,0.16}{##1}}}
\expandafter\def\csname PY@tok@nb\endcsname{\def\PY@tc##1{\textcolor[rgb]{0.00,0.50,0.00}{##1}}}
\expandafter\def\csname PY@tok@nc\endcsname{\let\PY@bf=\textbf\def\PY@tc##1{\textcolor[rgb]{0.00,0.00,1.00}{##1}}}
\expandafter\def\csname PY@tok@nd\endcsname{\def\PY@tc##1{\textcolor[rgb]{0.67,0.13,1.00}{##1}}}
\expandafter\def\csname PY@tok@ne\endcsname{\let\PY@bf=\textbf\def\PY@tc##1{\textcolor[rgb]{0.82,0.25,0.23}{##1}}}
\expandafter\def\csname PY@tok@nf\endcsname{\def\PY@tc##1{\textcolor[rgb]{0.00,0.00,1.00}{##1}}}
\expandafter\def\csname PY@tok@si\endcsname{\let\PY@bf=\textbf\def\PY@tc##1{\textcolor[rgb]{0.73,0.40,0.53}{##1}}}
\expandafter\def\csname PY@tok@s2\endcsname{\def\PY@tc##1{\textcolor[rgb]{0.73,0.13,0.13}{##1}}}
\expandafter\def\csname PY@tok@nt\endcsname{\let\PY@bf=\textbf\def\PY@tc##1{\textcolor[rgb]{0.00,0.50,0.00}{##1}}}
\expandafter\def\csname PY@tok@nv\endcsname{\def\PY@tc##1{\textcolor[rgb]{0.10,0.09,0.49}{##1}}}
\expandafter\def\csname PY@tok@s1\endcsname{\def\PY@tc##1{\textcolor[rgb]{0.73,0.13,0.13}{##1}}}
\expandafter\def\csname PY@tok@dl\endcsname{\def\PY@tc##1{\textcolor[rgb]{0.73,0.13,0.13}{##1}}}
\expandafter\def\csname PY@tok@ch\endcsname{\let\PY@it=\textit\def\PY@tc##1{\textcolor[rgb]{0.25,0.50,0.50}{##1}}}
\expandafter\def\csname PY@tok@m\endcsname{\def\PY@tc##1{\textcolor[rgb]{0.40,0.40,0.40}{##1}}}
\expandafter\def\csname PY@tok@gp\endcsname{\let\PY@bf=\textbf\def\PY@tc##1{\textcolor[rgb]{0.00,0.00,0.50}{##1}}}
\expandafter\def\csname PY@tok@sh\endcsname{\def\PY@tc##1{\textcolor[rgb]{0.73,0.13,0.13}{##1}}}
\expandafter\def\csname PY@tok@ow\endcsname{\let\PY@bf=\textbf\def\PY@tc##1{\textcolor[rgb]{0.67,0.13,1.00}{##1}}}
\expandafter\def\csname PY@tok@sx\endcsname{\def\PY@tc##1{\textcolor[rgb]{0.00,0.50,0.00}{##1}}}
\expandafter\def\csname PY@tok@bp\endcsname{\def\PY@tc##1{\textcolor[rgb]{0.00,0.50,0.00}{##1}}}
\expandafter\def\csname PY@tok@c1\endcsname{\let\PY@it=\textit\def\PY@tc##1{\textcolor[rgb]{0.25,0.50,0.50}{##1}}}
\expandafter\def\csname PY@tok@fm\endcsname{\def\PY@tc##1{\textcolor[rgb]{0.00,0.00,1.00}{##1}}}
\expandafter\def\csname PY@tok@o\endcsname{\def\PY@tc##1{\textcolor[rgb]{0.40,0.40,0.40}{##1}}}
\expandafter\def\csname PY@tok@kc\endcsname{\let\PY@bf=\textbf\def\PY@tc##1{\textcolor[rgb]{0.00,0.50,0.00}{##1}}}
\expandafter\def\csname PY@tok@c\endcsname{\let\PY@it=\textit\def\PY@tc##1{\textcolor[rgb]{0.25,0.50,0.50}{##1}}}
\expandafter\def\csname PY@tok@mf\endcsname{\def\PY@tc##1{\textcolor[rgb]{0.40,0.40,0.40}{##1}}}
\expandafter\def\csname PY@tok@err\endcsname{\def\PY@bc##1{\setlength{\fboxsep}{0pt}\fcolorbox[rgb]{1.00,0.00,0.00}{1,1,1}{\strut ##1}}}
\expandafter\def\csname PY@tok@mb\endcsname{\def\PY@tc##1{\textcolor[rgb]{0.40,0.40,0.40}{##1}}}
\expandafter\def\csname PY@tok@ss\endcsname{\def\PY@tc##1{\textcolor[rgb]{0.10,0.09,0.49}{##1}}}
\expandafter\def\csname PY@tok@sr\endcsname{\def\PY@tc##1{\textcolor[rgb]{0.73,0.40,0.53}{##1}}}
\expandafter\def\csname PY@tok@mo\endcsname{\def\PY@tc##1{\textcolor[rgb]{0.40,0.40,0.40}{##1}}}
\expandafter\def\csname PY@tok@kd\endcsname{\let\PY@bf=\textbf\def\PY@tc##1{\textcolor[rgb]{0.00,0.50,0.00}{##1}}}
\expandafter\def\csname PY@tok@mi\endcsname{\def\PY@tc##1{\textcolor[rgb]{0.40,0.40,0.40}{##1}}}
\expandafter\def\csname PY@tok@kn\endcsname{\let\PY@bf=\textbf\def\PY@tc##1{\textcolor[rgb]{0.00,0.50,0.00}{##1}}}
\expandafter\def\csname PY@tok@cpf\endcsname{\let\PY@it=\textit\def\PY@tc##1{\textcolor[rgb]{0.25,0.50,0.50}{##1}}}
\expandafter\def\csname PY@tok@kr\endcsname{\let\PY@bf=\textbf\def\PY@tc##1{\textcolor[rgb]{0.00,0.50,0.00}{##1}}}
\expandafter\def\csname PY@tok@s\endcsname{\def\PY@tc##1{\textcolor[rgb]{0.73,0.13,0.13}{##1}}}
\expandafter\def\csname PY@tok@kp\endcsname{\def\PY@tc##1{\textcolor[rgb]{0.00,0.50,0.00}{##1}}}
\expandafter\def\csname PY@tok@w\endcsname{\def\PY@tc##1{\textcolor[rgb]{0.73,0.73,0.73}{##1}}}
\expandafter\def\csname PY@tok@kt\endcsname{\def\PY@tc##1{\textcolor[rgb]{0.69,0.00,0.25}{##1}}}
\expandafter\def\csname PY@tok@sc\endcsname{\def\PY@tc##1{\textcolor[rgb]{0.73,0.13,0.13}{##1}}}
\expandafter\def\csname PY@tok@sb\endcsname{\def\PY@tc##1{\textcolor[rgb]{0.73,0.13,0.13}{##1}}}
\expandafter\def\csname PY@tok@sa\endcsname{\def\PY@tc##1{\textcolor[rgb]{0.73,0.13,0.13}{##1}}}
\expandafter\def\csname PY@tok@k\endcsname{\let\PY@bf=\textbf\def\PY@tc##1{\textcolor[rgb]{0.00,0.50,0.00}{##1}}}
\expandafter\def\csname PY@tok@se\endcsname{\let\PY@bf=\textbf\def\PY@tc##1{\textcolor[rgb]{0.73,0.40,0.13}{##1}}}
\expandafter\def\csname PY@tok@sd\endcsname{\let\PY@it=\textit\def\PY@tc##1{\textcolor[rgb]{0.73,0.13,0.13}{##1}}}

\def\PYZbs{\char`\\}
\def\PYZus{\char`\_}
\def\PYZob{\char`\{}
\def\PYZcb{\char`\}}
\def\PYZca{\char`\^}
\def\PYZam{\char`\&}
\def\PYZlt{\char`\<}
\def\PYZgt{\char`\>}
\def\PYZsh{\char`\#}
\def\PYZpc{\char`\%}
\def\PYZdl{\char`\$}
\def\PYZhy{\char`\-}
\def\PYZsq{\char`\'}
\def\PYZdq{\char`\"}
\def\PYZti{\char`\~}
% for compatibility with earlier versions
\def\PYZat{@}
\def\PYZlb{[}
\def\PYZrb{]}
\makeatother


    % Exact colors from NB
    \definecolor{incolor}{rgb}{0.0, 0.0, 0.5}
    \definecolor{outcolor}{rgb}{0.545, 0.0, 0.0}



    
    % Prevent overflowing lines due to hard-to-break entities
    \sloppy 
    % Setup hyperref package
    \hypersetup{
      breaklinks=true,  % so long urls are correctly broken across lines
      colorlinks=true,
      urlcolor=urlcolor,
      linkcolor=linkcolor,
      citecolor=citecolor,
      }
    % Slightly bigger margins than the latex defaults
    
    \geometry{verbose,tmargin=1in,bmargin=1in,lmargin=1in,rmargin=1in}
    
    

    \begin{document}
    
    
    \maketitle
    
    

    
    \section{Getting Started with
Equations}\label{getting-started-with-equations}

Equations are calculations in which one or more variables represent
unknown values. In this notebook, you'll learn some fundamental
techniques for solving simple equations.

\subsection{One Step Equations}\label{one-step-equations}

Consider the following equation:

\begin{equation}x + 16 = -25\end{equation}

The challenge here is to find the value for \textbf{x}, and to do this
we need to \emph{isolate the variable}. In this case, we need to get
\textbf{x} onto one side of the "=" sign, and all of the other values
onto the other side. To accomplish this we'll follow these rules: 1. Use
opposite operations to cancel out the values we don't want on one side.
In this case, the left side of the equation includes an addition of 16,
so we'll cancel that out by subtracting 16 and the left side of the
equation becomes \textbf{x + 16 - 16}. 2. Whatever you do to one side,
you must also do to the other side. In this case, we subtracted 16 from
the left side, so we must also subtract 16 from the right side of the
equation, which becomes \textbf{-25 - 16}. Our equation now looks like
this:

\begin{equation}x + 16 - 16 = -25 - 16\end{equation}

Now we can calculate the values on both side. On the left side, 16 - 16
is 0, so we're left with:

\begin{equation}x = -25 - 16\end{equation}

Which yields the result \textbf{-41}. Our equation is now solved, as you
can see here:

\begin{equation}x = -41\end{equation}

It's always good practice to verify your result by plugging the variable
value you've calculated into the original equation and ensuring that it
holds true. We can easily do that by using some simple Python code.

To verify the equation using Python code, place the cursor in the
following cell and then click the ►\textbar{} button in the toolbar.

    \begin{Verbatim}[commandchars=\\\{\}]
{\color{incolor}In [{\color{incolor}3}]:} \PY{n}{x} \PY{o}{=} \PY{o}{\PYZhy{}}\PY{l+m+mi}{41}
        \PY{n}{x} \PY{o}{+} \PY{l+m+mi}{16} \PY{o}{==} \PY{o}{\PYZhy{}}\PY{l+m+mi}{25}
\end{Verbatim}


\begin{Verbatim}[commandchars=\\\{\}]
{\color{outcolor}Out[{\color{outcolor}3}]:} True
\end{Verbatim}
            
    \subsection{Two-Step Equations}\label{two-step-equations}

The previous example was fairly simple - you could probably work it out
in your head. So what about something a little more complex?

\begin{equation}3x - 2 = 10 \end{equation}

As before, we need to isolate the variable \textbf{x}, but this time
we'll do it in two steps. The first thing we'll do is to cancel out the
\emph{constants}. A constant is any number that stands on its own, so in
this case the 2 that we're subtracting on the left side is a constant.
We'll use an opposite operation to cancel it out on the left side, so
since the current operation is to subtract 2, we'll add 2; and of course
whatever we do on the left side we also need to do on the right side, so
after the first step, our equation looks like this:

\begin{equation}3x - 2 + 2 = 10 + 2 \end{equation}

Now the -2 and +2 on the left cancel one another out, and on the right
side, 10 + 2 is 12; so the equation is now:

\begin{equation}3x = 12 \end{equation}

OK, time for step two - we need to deal with the \emph{coefficients} - a
coefficient is a number that is applied to a variable. In this case, our
expression on the left is 3x, which means x multiplied by 3; so we can
apply the opposite operation to cancel it out as long as we do the same
to the other side, like this:

\begin{equation}\frac{3x}{3} = \frac{12}{3} \end{equation}

3x ÷ 3 is x, so we've now isolated the variable

\begin{equation}x = \frac{12}{3} \end{equation}

And we can calculate the result as 12/3 which is \textbf{4}:

\begin{equation}x = 4 \end{equation}

Let's verify that result using Python:

    \begin{Verbatim}[commandchars=\\\{\}]
{\color{incolor}In [{\color{incolor}4}]:} \PY{n}{x} \PY{o}{=} \PY{l+m+mi}{4}
        \PY{l+m+mi}{3}\PY{o}{*}\PY{n}{x} \PY{o}{\PYZhy{}} \PY{l+m+mi}{2} \PY{o}{==} \PY{l+m+mi}{10}
\end{Verbatim}


\begin{Verbatim}[commandchars=\\\{\}]
{\color{outcolor}Out[{\color{outcolor}4}]:} True
\end{Verbatim}
            
    \subsection{Combining Like Terms}\label{combining-like-terms}

Like terms are elements of an expression that relate to the same
variable or constant (with the same \emph{order} or \emph{exponential},
which we'll discuss later). For example, consider the following
equation:

\begin{equation}\textbf{5x} + 1 \textbf{- 2x} = 22 \end{equation}

In this equation, the left side includes the terms \textbf{5x} and
\textbf{- 2x}, both of which represent the variable \textbf{x}
multiplied by a coefficent. Note that we include the sign (+ or -) in
front of the value.

We can rewrite the equation to combine these like terms:

\begin{equation}\textbf{5x - 2x} + 1 = 22 \end{equation}

We can then simply perform the necessary operations on the like terms to
consolidate them into a single term:

\begin{equation}\textbf{3x} + 1 = 22 \end{equation}

Now, we can solve this like any other two-step equation. First we'll
remove the constants from the left side - in this case, there's a
constant expression that adds 1, so we'll use the opposite operation to
remove it and do the same on the other side:

\begin{equation}3x + 1 - 1 = 22 - 1 \end{equation}

That gives us:

\begin{equation}3x = 21 \end{equation}

Then we'll deal with the coefficients - in this case x is multiplied by
3, so we'll divide by 3 on boths sides to remove that:

\begin{equation}\frac{3x}{3} = \frac{21}{3} \end{equation}

This give us our answer:

\begin{equation}x = 7 \end{equation}

    \begin{Verbatim}[commandchars=\\\{\}]
{\color{incolor}In [{\color{incolor} }]:} \PY{n}{x} \PY{o}{=} \PY{l+m+mi}{7}
        \PY{l+m+mi}{5}\PY{o}{*}\PY{n}{x} \PY{o}{+} \PY{l+m+mi}{1} \PY{o}{\PYZhy{}} \PY{l+m+mi}{2}\PY{o}{*}\PY{n}{x} \PY{o}{==} \PY{l+m+mi}{22}
\end{Verbatim}


    \subsection{Working with Fractions}\label{working-with-fractions}

Some of the steps in solving the equations above have involved working
wth fractions - which in themselves are actually just division
operations. Let's take a look at an example of an equation in which our
variable is defined as a fraction:

\begin{equation}\frac{x}{3} + 1 = 16 \end{equation}

We follow the same approach as before, first removing the constants from
the left side - so we'll subtract 1 from both sides.

\begin{equation}\frac{x}{3} = 15 \end{equation}

Now we need to deal with the fraction on the left so that we're left
with just \textbf{x}. The fraction is x/3 which is another way of saying
\emph{x divided by 3}, so we can apply the opposite operation to both
sides. In this case, we need to multiply both sides by the denominator
under our variable, which is 3. To make it easier to work with a term
that contains fractions, we can express whole numbers as fractions with
a denominator of 1; so on the left, we can express 3 as 3/1 and multiply
it with x/3. Note that the notation for mutiplication is a \textbf{•}
symbol rather than the standard \emph{x} multiplication operator (which
would cause confusion with the variable \textbf{x}) or the asterisk
symbol used by most programming languages.

\begin{equation}\frac{3}{1} \cdot \frac{x}{3} = 15 \cdot 3 \end{equation}

This gives us the following result:

\begin{equation}x = 45 \end{equation}

Let's verify that with some Python code:

    \begin{Verbatim}[commandchars=\\\{\}]
{\color{incolor}In [{\color{incolor} }]:} \PY{n}{x} \PY{o}{=} \PY{l+m+mi}{45}
        \PY{n}{x}\PY{o}{/}\PY{l+m+mi}{3} \PY{o}{+} \PY{l+m+mi}{1} \PY{o}{==} \PY{l+m+mi}{16}
\end{Verbatim}


    Let's look at another example, in which the variable is a whole number,
but its coefficient is a fraction:

\begin{equation}\frac{2}{5}x + 1 = 11 \end{equation}

As usual, we'll start by removing the constants from the variable
expression; so in this case we need to subtract 1 from both sides:

\begin{equation}\frac{2}{5}x = 10 \end{equation}

Now we need to cancel out the fraction. The expression equates to
two-fifths times x, so the opposite operation is to divide by 2/5; but a
simpler way to do this with a fraction is to multiply it by its
\emph{reciprocal}, which is just the inverse of the fraction, in this
case 5/2. Of course, we need to do this to both sides:

\begin{equation}\frac{5}{2} \cdot \frac{2}{5}x = \frac{10}{1} \cdot \frac{5}{2} \end{equation}

That gives us the following result:

\begin{equation}x = \frac{50}{2} \end{equation}

Which we can simplify to:

\begin{equation}x = 25 \end{equation}

We can confirm that with Python:

    \begin{Verbatim}[commandchars=\\\{\}]
{\color{incolor}In [{\color{incolor} }]:} \PY{n}{x} \PY{o}{=} \PY{l+m+mi}{25}
        \PY{l+m+mi}{2}\PY{o}{/}\PY{l+m+mi}{5} \PY{o}{*} \PY{n}{x} \PY{o}{+} \PY{l+m+mi}{1} \PY{o}{==}\PY{l+m+mi}{11}
\end{Verbatim}


    \subsection{Equations with Variables on Both
Sides}\label{equations-with-variables-on-both-sides}

So far, all of our equations have had a variable term on only one side.
However, variable terms can exist on both sides.

Consider this equation:

\begin{equation}3x + 2 = 5x - 1 \end{equation}

This time, we have terms that include \textbf{x} on both sides. Let's
take exactly the same approach to solving this kind of equation as we
did for the previous examples. First, let's deal with the constants by
adding 1 to both sides. That gets rid of the -1 on the right:

\begin{equation}3x + 3 = 5x \end{equation}

Now we can eliminate the variable expression from one side by
subtracting 3x from both sides. That gets rid of the 3x on the left:

\begin{equation}3 = 2x \end{equation}

Next, we can deal with the coefficient by dividing both sides by 2:

\begin{equation}\frac{3}{2} = x \end{equation}

Now we've isolated x. It looks a little strange because we usually have
the variable on the left side, so if it makes you more comfortable you
can simply reverse the equation:

\begin{equation}x = \frac{3}{2} \end{equation}

Finally, this answer is correct as it is; but 3/2 is an improper
fraction. We can simplify it to:

\begin{equation}x = 1\frac{1}{2} \end{equation}

So x is 11/2 (which is of course 1.5 in decimal notation). Let's check
it in Python:

    \begin{Verbatim}[commandchars=\\\{\}]
{\color{incolor}In [{\color{incolor}5}]:} \PY{n}{x} \PY{o}{=} \PY{l+m+mf}{1.5}
        \PY{l+m+mi}{3}\PY{o}{*}\PY{n}{x} \PY{o}{+} \PY{l+m+mi}{2} \PY{o}{==} \PY{l+m+mi}{5}\PY{o}{*}\PY{n}{x} \PY{o}{\PYZhy{}}\PY{l+m+mi}{1}
\end{Verbatim}


\begin{Verbatim}[commandchars=\\\{\}]
{\color{outcolor}Out[{\color{outcolor}5}]:} True
\end{Verbatim}
            
    \subsection{Using the Distributive
Property}\label{using-the-distributive-property}

The distributive property is a mathematical law that enables us to
distribute the same operation to terms within parenthesis. For example,
consider the following equation:

\begin{equation}\textbf{4(x + 2)} + \textbf{3(x - 2)} = 16 \end{equation}

The equation includes two operations in parenthesis: \textbf{4(\emph{x}
+ 2)} and \textbf{3(\emph{x} - 2)}. Each of these operations consists of
a constant by which the contents of the parenthesis must be multipled:
for example, 4 times (\emph{x} + 2). The distributive property means
that we can achieve the same result by multiplying each term in the
parenthesis and adding the results, so for the first parenthetical
operation, we can multiply 4 by \emph{x} and add it to 4 times +2; and
for the second parenthetical operation, we can calculate 3 times
\emph{x} + 3 times -2). Note that the constants in the parenthesis
include the sign (+ or -) that preceed them:

\begin{equation}4x + 8 + 3x - 6 = 16 \end{equation}

Now we can group our like terms:

\begin{equation}7x + 2 = 16 \end{equation}

Then we move the constant to the other side:

\begin{equation}7x = 14 \end{equation}

And now we can deal with the coefficient:

\begin{equation}\frac{7x}{7} = \frac{14}{7} \end{equation}

Which gives us our anwer:

\begin{equation}x = 2 \end{equation}

Here's the original equation with the calculated value for \emph{x} in
Python:

    \begin{Verbatim}[commandchars=\\\{\}]
{\color{incolor}In [{\color{incolor}6}]:} \PY{n}{x} \PY{o}{=} \PY{l+m+mi}{2}
        \PY{l+m+mi}{4}\PY{o}{*}\PY{p}{(}\PY{n}{x} \PY{o}{+} \PY{l+m+mi}{2}\PY{p}{)} \PY{o}{+} \PY{l+m+mi}{3}\PY{o}{*}\PY{p}{(}\PY{n}{x} \PY{o}{\PYZhy{}} \PY{l+m+mi}{2}\PY{p}{)} \PY{o}{==} \PY{l+m+mi}{16}
\end{Verbatim}


\begin{Verbatim}[commandchars=\\\{\}]
{\color{outcolor}Out[{\color{outcolor}6}]:} True
\end{Verbatim}
            

    % Add a bibliography block to the postdoc
    
    
    
    \end{document}
